\documentclass[12pt,a4paper]{report}
\usepackage[utf8]{inputenc}
\usepackage[T1]{fontenc}
\usepackage[brazil]{babel}
\usepackage{indentfirst}
\usepackage{geometry}
\usepackage{titlesec}
\usepackage{enumitem}

% Configuração da página
\geometry{
    a4paper,
    top=2.5cm,
    bottom=2.5cm,
    left=3cm,
    right=2cm
}

\begin{document}

\begin{center}
{\Large\textbf{Referências Bibliográficas por Volume}}
\vspace{1cm}
\end{center}

\chapter*{Volume 1: Imperialismo e Transferência Tecnológica (1880-1929)}

\section*{Documentos Base}
\begin{enumerate}[leftmargin=*]
    \item COMISSÃO MISTA BRASIL-ESTADOS UNIDOS PARA DESENVOLVIMENTO ECONÔMICO. \textbf{O Desenvolvimento Econômico do Brasil}: Relatório da Comissão Mista Brasil-Estados Unidos. Rio de Janeiro: Imprensa Nacional, 1949. 2 v.
    
    \item LIGHT AND POWER CO. \textbf{Annual Report}. Toronto: Light and Power Co., 1900-1929. Relatórios anuais.
    
    \item AMERICAN \& FOREIGN POWER COMPANY. \textbf{Report to the Securities and Exchange Commission}. New York: SEC Archives, 1923-1929.
\end{enumerate}

\section*{Literatura Especializada}
\begin{enumerate}[leftmargin=*]
    \item ARMSTRONG, Christopher; NELLES, Henry Vivian. \textbf{Southern exposure}: Canadian promoters in Latin America and the Caribbean, 1896-1930. Toronto: University of Toronto Press, 1988.
    
    \item BRANCO, Catullo. \textbf{Energia elétrica e capital estrangeiro no Brasil}. São Paulo: Alfa-Omega, 1975.
    
    \item SAES, Alexandre Macchione. \textbf{Conflitos do capital}: Light versus CBEE na formação do capitalismo brasileiro (1898-1927). Bauru: EDUSC, 2010.
\end{enumerate}

\chapter*{Volume 2: Estado e Desenvolvimento Tecnológico (1930-1960)}

\section*{Documentos Base}
\begin{enumerate}[leftmargin=*]
    \item BRASIL. \textbf{Código de Águas}. Decreto nº 24.643, de 10 de julho de 1934. Rio de Janeiro: Imprensa Nacional, 1934.
    
    \item MISSÃO COOKE. \textbf{A Missão Cooke no Brasil}: Relatório dirigido ao Presidente dos Estados Unidos da América pela Missão Técnica Americana enviada ao Brasil. Rio de Janeiro: FGV, 1949.
    
    \item CHESF. \textbf{Relatório da Diretoria}. Recife: CHESF, 1948-1960. Relatórios anuais.
\end{enumerate}

\section*{Literatura Especializada}
\begin{enumerate}[leftmargin=*]
    \item DRAIBE, Sônia. \textbf{Rumos e metamorfoses}: um estudo sobre a constituição do Estado e as alternativas da industrialização no Brasil, 1930-1960. Rio de Janeiro: Paz e Terra, 1985.
    
    \item LIMA, José Luiz. \textbf{Estado e energia no Brasil}: o setor elétrico no Brasil: das origens à criação da Eletrobrás (1890-1962). São Paulo: IPE/USP, 1984.
    
    \item TENDLER, Judith. \textbf{Electric power in Brazil}: entrepreneurship in the public sector. Cambridge: Harvard University Press, 1968.
\end{enumerate}

\chapter*{Volume 3: Territorialização e Sistemas Técnicos (1961-1970)}

\section*{Documentos Base}
\begin{enumerate}[leftmargin=*]
    \item BRASIL. Ministério das Minas e Energia. \textbf{Plano Nacional de Eletrificação}. Brasília: MME, 1965.
    
    \item ELETROBRÁS. \textbf{Relatório de Atividades}. Rio de Janeiro: Eletrobrás, 1962-1970. Relatórios anuais.
    
    \item CANAMBRA ENGINEERING CONSULTANTS. \textbf{Power Study of South Central Brazil}. Rio de Janeiro: Canambra, 1966.
\end{enumerate}

\section*{Literatura Especializada}
\begin{enumerate}[leftmargin=*]
    \item DIAS, Renato Feliciano. \textbf{Panorama do setor de energia elétrica no Brasil}. Rio de Janeiro: Centro da Memória da Eletricidade no Brasil, 1988.
    
    \item SANTOS, Milton; SILVEIRA, María Laura. \textbf{O Brasil}: território e sociedade no início do século XXI. Rio de Janeiro: Record, 2001.
    
    \item VIANNA, Eduardo da Cunha. \textbf{CHESF}: 35 anos de história. Recife: CHESF, 1983.
\end{enumerate}

\chapter*{Volume 4: Grandes Projetos e Inovação (1971-1980)}

\section*{Documentos Base}
\begin{enumerate}[leftmargin=*]
    \item ITAIPU BINACIONAL. \textbf{Relatório Anual}. Foz do Iguaçu: Itaipu, 1974-1980.
    
    \item ELETROBRÁS. \textbf{Plano de Expansão do Setor Elétrico}. Rio de Janeiro: Eletrobrás, 1975.
    
    \item CEPEL. \textbf{Relatório de Atividades}. Rio de Janeiro: CEPEL, 1974-1980.
\end{enumerate}

\section*{Literatura Especializada}
\begin{enumerate}[leftmargin=*]
    \item COTRIM, John. \textbf{A história de Itaipu}. Curitiba: Itaipu Binacional, 1999.
    
    \item LEITE, Antônio Dias. \textbf{A energia do Brasil}. 2. ed. Rio de Janeiro: Elsevier, 2007.
    
    \item MIELNIK, Otávio; NEVES, C. C. \textbf{Características da estrutura de produção de energia hidroelétrica no Brasil}. In: ROSA, L. P. et al. (org.). Impactos de grandes projetos hidrelétricos e nucleares. São Paulo: Marco Zero, 1988.
\end{enumerate}

\chapter*{Volume 5: Novos Paradigmas Tecnológicos (1981-1988)}

\section*{Documentos Base}
\begin{enumerate}[leftmargin=*]
    \item BRASIL. Ministério das Minas e Energia. \textbf{Plano Nacional de Energia Elétrica 1987/2010}. Brasília: MME/Eletrobrás, 1987.
    
    \item CEPEL. \textbf{Relatório do Programa de P\&D}. Rio de Janeiro: CEPEL, 1981-1988.
    
    \item ELETROBRÁS. \textbf{Revisão Institucional do Setor Elétrico}. Rio de Janeiro: Eletrobrás, 1988.
\end{enumerate}

\section*{Literatura Especializada}
\begin{enumerate}[leftmargin=*]
    \item ARAÚJO, João Lizardo de. \textbf{A questão do investimento no setor elétrico brasileiro}: reforma e crise. Nova Economia, v. 11, n. 1, p. 77-96, 2001.
    
    \item GOMES, Antonio C. S. et al. \textbf{O setor elétrico}. In: BNDES. BNDES 50 Anos: histórias setoriais. Rio de Janeiro: BNDES, 2002.
    
    \item ROSA, Luiz Pinguelli; SIGAUD, Lygia; MIELNIK, Otávio (org.). \textbf{Impactos de grandes projetos hidrelétricos e nucleares}: aspectos econômicos, tecnológicos, ambientais e sociais. São Paulo: Marco Zero, 1988.
\end{enumerate}

\chapter*{Referências Gerais (Utilizadas em Múltiplos Volumes)}

\begin{enumerate}[leftmargin=*]
    \item CENTRO DA MEMÓRIA DA ELETRICIDADE NO BRASIL. \textbf{Panorama do setor de energia elétrica no Brasil}. Rio de Janeiro: Centro da Memória da Eletricidade no Brasil, 1988.
    
    \item DIAS, Renato Feliciano (coord.). \textbf{A Eletrobrás e a história do setor de energia elétrica no Brasil}. Rio de Janeiro: Centro da Memória da Eletricidade no Brasil, 1995.
    
    \item FURTADO, Celso. \textbf{Formação econômica do Brasil}. 34. ed. São Paulo: Companhia das Letras, 2007.
    
    \item LEITE, Antônio Dias. \textbf{A energia do Brasil}. 2. ed. Rio de Janeiro: Elsevier, 2007.
    
    \item LIMA, José Luiz. \textbf{Políticas de governo e desenvolvimento do setor de energia elétrica}: do Código de Águas à crise dos anos 80 (1934-1984). Rio de Janeiro: Centro da Memória da Eletricidade no Brasil, 1995.
    
    \item SANTOS, Milton. \textbf{A natureza do espaço}: técnica e tempo, razão e emoção. 4. ed. São Paulo: Edusp, 2006.
    
    \item SILVA, Ligia Maria Osório. \textbf{A apropriação territorial na Primeira República}. In: SILVA, S. S.; SZMRECSÁNYI, T. (org.). História econômica da Primeira República. 2. ed. São Paulo: Hucitec, 2002.
\end{enumerate}

\end{document} 