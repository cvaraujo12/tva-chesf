\documentclass[12pt,a4paper]{report}
\usepackage[utf8]{inputenc}
\usepackage[T1]{fontenc}
\usepackage{titlesec}
\usepackage{indentfirst}
\usepackage[brazil]{babel}
\usepackage{hyperref}
\usepackage{tocloft}
\usepackage{geometry}
\usepackage{enumitem}
\usepackage{fancyhdr}

% Configuração da página
\geometry{
    a4paper,
    top=2.5cm,
    bottom=2.5cm,
    left=3cm,
    right=2cm
}

% Configuração dos estilos
\titleformat{\part}{\normalfont\Huge\bfseries}{\thepart}{1em}{}
\titleformat{\chapter}{\normalfont\LARGE\bfseries}{\thechapter}{1em}{}
\titleformat{\section}{\normalfont\Large\bfseries}{\thesection}{1em}{}
\titleformat{\subsection}{\normalfont\large\bfseries}{\thesubsection}{1em}{}

% Configuração do sumário
\renewcommand{\cfttoctitlefont}{\hfill\LARGE\bfseries}
\renewcommand{\cftaftertoctitle}{\hfill}
\renewcommand{\cftpartfont}{\Large\bfseries}
\renewcommand{\cftchapfont}{\bfseries}
\renewcommand{\cftsecfont}{\normalfont}
\renewcommand{\cftsubsecfont}{\normalfont}

% Configuração do cabeçalho e rodapé
\pagestyle{fancy}
\fancyhf{}
\fancyhead[L]{\slshape\nouppercase{\leftmark}}
\fancyhead[R]{\thepage}
\renewcommand{\headrulewidth}{0.4pt}

\begin{document}

% Página de título
\begin{titlepage}
\begin{center}
\vspace*{2cm}
{\Huge\textbf{Desenvolvimento Tecnológico do\\Setor Elétrico Brasileiro}}
\vspace{1.5cm}

{\Large (1880-1988)}
\vspace{2cm}

{\large\textbf{Índice Geral da Coletânea}}
\vfill

{\large\today}
\end{center}
\end{titlepage}

% Sumário
\tableofcontents
\newpage

% Parte I
\part{Documentação de Apoio}

\chapter{Documentos Base}

\section{Base Bibliográfica 1 (BB1)}
\subsection{Relatório Geral CMBEU}
\begin{itemize}[leftmargin=*]
    \item Primeiro Tomo
    \item Segundo Tomo
\end{itemize}

\subsection{Projetos Energia}
\begin{itemize}[leftmargin=*]
    \item Volume 10
    \item Documentação técnica
\end{itemize}

\section{Base Bibliográfica 2 (BB2)}
\subsection{Relatório Geral Cooke}
\begin{itemize}[leftmargin=*]
    \item Análise do desenvolvimento
    \item Recomendações técnicas
    \item Aspectos energéticos
\end{itemize}

\section{Base Bibliográfica 3 (BB3)}
\subsection{Artigos Acadêmicos}
\begin{itemize}[leftmargin=*]
    \item Desenvolvimento territorial
    \item Aspectos tecnológicos
    \item Análises históricas
\end{itemize}

\section{Base Bibliográfica 4 (BB4)}
\subsection{Documentos Contemporâneos}
\begin{itemize}[leftmargin=*]
    \item Síntese de documentos
    \item Análises finais
    \item Relatórios técnicos
\end{itemize}

\section{Estudos Complementares}
\begin{itemize}[leftmargin=*]
    \item Projeto Final
    \item Período Econômico Clássico CHESF
    \item Relatórios técnicos específicos
\end{itemize}

\chapter{Documentos Metodológicos}

\section{Estrutura metodológica}
\begin{itemize}[leftmargin=*]
    \item Abordagem de pesquisa
    \item Procedimentos analíticos
    \item Protocolos de investigação
\end{itemize}

\section{Procedimentos de pesquisa}
\begin{itemize}[leftmargin=*]
    \item Coleta de dados
    \item Análise documental
    \item Sistematização
\end{itemize}

\section{Protocolos de análise}
\begin{itemize}[leftmargin=*]
    \item Critérios técnicos
    \item Parâmetros avaliativos
    \item Métricas de análise
\end{itemize}

\chapter{Linhas do Tempo}

\section{Setor Elétrico Nacional (1880-1988)}
\begin{itemize}[leftmargin=*]
    \item Evolução tecnológica
    \item Marcos institucionais
    \item Transformações setoriais
\end{itemize}

\section{Paulo Afonso (1945-1988)}
\begin{itemize}[leftmargin=*]
    \item Desenvolvimento do complexo
    \item Aspectos militares
    \item Inovações técnicas
\end{itemize}

\section{Cronologia Tecnológica}
\begin{itemize}[leftmargin=*]
    \item Avanços técnicos
    \item Inovações setoriais
    \item Mudanças paradigmáticas
\end{itemize}

% Parte II
\part{Volumes da Coletânea}

\chapter{Volume 1: Imperialismo e Transferência Tecnológica (1880-1929)}

\section{Introdução}
\begin{itemize}[leftmargin=*]
    \item Contexto histórico
    \item Marco teórico-conceitual
    \item Metodologia
\end{itemize}

\section{Primórdios da Eletrificação}
\begin{itemize}[leftmargin=*]
    \item Primeiras Experiências Tecnológicas
    \item Tecnologias Importadas
    \item Formação Técnica Inicial
\end{itemize}

\section{Era das Concessionárias Estrangeiras}
\begin{itemize}[leftmargin=*]
    \item Light and Power
    \item AMFORP
    \item Transferência Tecnológica
\end{itemize}

\section{Base Industrial Nacional}
\begin{itemize}[leftmargin=*]
    \item Primeiros Fabricantes
    \item Capacitação Local
\end{itemize}

\chapter{Volume 2: Estado e Desenvolvimento Tecnológico (1930-1960)}

\section{Introdução}
\begin{itemize}[leftmargin=*]
    \item Contexto político-econômico
    \item Mudanças institucionais
    \item Nova fase tecnológica
\end{itemize}

[continua com o mesmo padrão para os volumes 3, 4 e 5...]

% Parte III
\part{Elementos Complementares}

\chapter{Conclusão Geral}
\section{Síntese histórica}
\section{Evolução tecnológica}
\section{Legado setorial}
\section{Contribuições originais}

\chapter{Apêndices}

\section{Evolução dos Sistemas}
\begin{itemize}[leftmargin=*]
    \item Tecnologias de geração
    \item Sistemas de transmissão
    \item Redes de distribuição
    \item Operação integrada
\end{itemize}

\section{Centros de P\&D}
\begin{itemize}[leftmargin=*]
    \item Estrutura do CEPEL
    \item Laboratórios setoriais
    \item Universidades
    \item Cooperação internacional
\end{itemize}

\section{Base Industrial}
\begin{itemize}[leftmargin=*]
    \item Fabricantes nacionais
    \item Desenvolvimento tecnológico
    \item Normalização técnica
    \item Cadeia produtiva
\end{itemize}

\chapter{Documentação Complementar}

\section{Referências Bibliográficas}
\begin{itemize}[leftmargin=*]
    \item Volume 1: Imperialismo e Transferência Tecnológica
    \item Volume 2: Estado e Desenvolvimento Tecnológico
    \item Volume 3: Territorialização e Sistemas Técnicos
    \item Volume 4: Grandes Projetos e Inovação
    \item Volume 5: Novos Paradigmas Tecnológicos
    \item Referências Gerais
\end{itemize}

\section{Fontes Consultadas}
\begin{itemize}[leftmargin=*]
    \item Documentos e Relatórios Oficiais
    \item Acervos e Arquivos Históricos
    \item Bases de Dados e Repositórios Digitais
    \item Periódicos Históricos
    \item Acervos Institucionais
    \item Bases de Dados Online
\end{itemize}

\section{Índices Remissivos}
\begin{itemize}[leftmargin=*]
    \item Índice onomástico
    \item Índice de instituições
    \item Índice de localidades
    \item Índice de temas
\end{itemize}

\end{document} 